\lhead{\textit{Anexos}}
\chead{}
\rhead{\thepage}



\chapter*{Anexos}
\addcontentsline{toc}{chapter}{Anexos}


En ésta sección se presentan los anexos utilizados para el desarrollo del  proyecto: código fuente de los algoritmos implementados y de otras herramientas complementarias.

\newpage

\section*{Simulación básica en \textit{CloudSim}}
\addcontentsline{toc}{section}{Simulación Básica en \textit{CloudSim}}

\begin{lstlisting}
/*
* Title:        CloudSim Toolkit
* Description:  CloudSim (Cloud Simulation) Toolkit for Modeling and Simulation
*               of Clouds
* Licence:      GPL - http://www.gnu.org/copyleft/gpl.html
*
* Copyright (c) 2009, The University of Melbourne, Australia
*/
package FCFS;

import java.util.*;

import common.CloudletUtilities;
import org.cloudbus.cloudsim.*;
import org.cloudbus.cloudsim.core.CloudSim;
import common.VirtualMachineCreator;
import common.CloudletCreator;
import common.DataCenterCreator;
/**
* FCFS Task scheduling
* @author
* Canul, Cindy
* Kumul, Cristian
* Peraza, Jonathan
*/
public class FCFS {

//Lista de tareas
private static List<Cloudlet> cloudletList;

// Lista de maquinas virtuales
private static List<Vm> vmlist;


public static List<Cloudlet> main(String[] args) {

Log.printLine("Iniciando FCFS...");

try {
// Primer paso: Se inicializa CloudSim

int num_user = 1;   // numero de usuarios
Calendar calendar = Calendar.getInstance();
boolean trace_flag = false;  // caracteristica de seguimiento de eventos desactivada

// La libreria CloudSim es iniciada
CloudSim.init(num_user, calendar, trace_flag);

// Segundo paso: Crear el centro de datos
//Los centros de datos son los proveedores de recursos en CloudSim . Necesitamos a la lista uno de ellos para ejecutar una simulaci\'on CloudSim

Datacenter datacenter0 =  new DataCenterCreator().createDatacenter("Datacenter_0");


//Tercer paso: Crear un broker
//DatacenterBroker broker = createBroker();
FcfsBroker broker = createBroker();
int brokerId = broker.getId();

//Cuarto paso: Crear las maquinas virtuales
//vmlist = new VmsCreator().createVirtualMachines(brokerId);
vmlist = new VirtualMachineCreator().createVM(brokerId, 15, 0);
//se crean las tareas
cloudletList = new CloudletCreator().createTasks(brokerId, 500, 0);

Log.printLine("Initial Cloudlet :"+cloudletList.size());

//se envia la lista de maquinas virtuales al broker
broker.submitVmList(vmlist);


//Quinto paso: Enviar las tareas al broker

broker.submitCloudletList(cloudletList);


//llamar a la funcion del broker encargada de calendarizar las tareas a las m\'aquinas virtuales
broker.scheduleTaskstoVms();


//Sexto paso: Iniciar la simulacion
CloudSim.startSimulation();


// Paso final: Obtener el listado de las tareas completadas
List<Cloudlet> newList = broker.getCloudletReceivedList();

CloudSim.stopSimulation();

CloudletUtilities.printCloudletList(newList);

return newList;
//Log.printLine("FCFS finished!");
}
catch (Exception e) {
e.printStackTrace();
Log.printLine("The simulation has been terminated due to an unexpected error");
}
return null;
}



// funcion que crea un broker del tipo especifico para cada algoritmo
private static FcfsBroker createBroker(){

FcfsBroker broker = null;
try {
broker = new FcfsBroker("Broker");
} catch (Exception e) {
e.printStackTrace();
return null;
}
return broker;
}

}

\end{lstlisting}

\newpage

\section*{\textit{Cloud Brokers}}
\addcontentsline{toc}{section}{\textit{Cloud Brokers}}
\subsection*{\textit{FCFS Broker}}

\begin{lstlisting}
package FCFS;

import PSO.Particle;
import org.cloudbus.cloudsim.DatacenterBroker;
import org.cloudbus.cloudsim.Cloudlet;
import org.cloudbus.cloudsim.Vm;
import org.cloudbus.cloudsim.Host;
import PSO.FitnessFunction;

/**
* FCFS Task scheduling
* @author
* Canul, Cindy
* Kumul, Cristian
* Peraza, Jonathan
*/
public class FcfsBroker extends DatacenterBroker {

public FcfsBroker(String name) throws Exception {
super(name);
// TODO Auto-generated constructor stub
}

//Funcion de calendarizacion
public void scheduleTaskstoVms(){
int reqTasks=cloudletList.size();
int reqVms=vmList.size();

System.out.println("\n\tFCFS Broker Schedules\n");
for(int i=0;i<reqTasks;i++){
//Asignacion de las tareas a las maquinas virtuales
bindCloudletToVm(i, (i%reqVms));
System.out.println("Task"+cloudletList.get(i).getCloudletId()+" is bound with VM"+vmList.get(i%reqVms).getId() +  "tam: " + cloudletList.get(i).getCloudletFileSize());
}

System.out.println("\n");
}
}
\end{lstlisting}


\subsection*{\textit{Max-Min Broker}}
\begin{lstlisting}

package Maxmin;


import org.cloudbus.cloudsim.DatacenterBroker;
import org.cloudbus.cloudsim.lists.CloudletList;
/**
* FCFS Task scheduling
* @author
* Canul, Cindy
* Kumul, Cristian
* Peraza, Jonathan
*/
public class MaxminBroker extends DatacenterBroker {

public MaxminBroker(String name) throws Exception {
super(name);
// TODO Auto-generated constructor stub
}

//funcion de calendarizacion
public void scheduleTaskstoVms(){
int reqTasks=cloudletList.size();
int reqVms=vmList.size();

System.out.println("\n\tMaxmin Broker Schedules\n");

CloudletList.revert(cloudletList);

for(int i=0;i<reqTasks;i++){
bindCloudletToVm(i, (i%reqVms));
System.out.println("Task"+cloudletList.get(i).getCloudletId()+" is bound with VM"+vmList.get(i%reqVms).getId() +  "tam: " + cloudletList.get(i).getCloudletLength());
}

System.out.println("\n");
}
}

\end{lstlisting}



\subsection*{\textit{Min-Min Broker}}
\begin{lstlisting}
package Minmin;


import org.cloudbus.cloudsim.DatacenterBroker;
import org.cloudbus.cloudsim.Log;
import org.cloudbus.cloudsim.lists.CloudletList;

/**
* FCFS Task scheduling
* @author
* Canul, Cindy
* Kumul, Cristian
* Peraza, Jonathan
*/

public class MinminBroker extends DatacenterBroker {

public MinminBroker(String name) throws Exception {
super(name);
// TODO Auto-generated constructor stub
}

//funcion de calendarizacion
public void scheduleTaskstoVms(){
int numTareas = cloudletList.size();
int numVirtualM = vmList.size();

System.out.println("\n\tMinmin Broker Schedules\n");

CloudletList.sort(cloudletList);

//Asignamos las tareas a cada maquina virtual de manera que se atiendan primero las de menor tama-o
for(int i=0;i<numTareas;i++){
bindCloudletToVm(i, (i%numVirtualM));
System.out.println("Task" + cloudletList.get(i).getCloudletId() + " is bound with VM" + vmList.get(i % numVirtualM).getId() + " tam: " + cloudletList.get(i).getCloudletLength());
}

}
}

\end{lstlisting}

\subsection*{\textit{Round Robin Broker}}
\begin{lstlisting}
package RoundRobin;

import org.cloudbus.cloudsim.DatacenterBroker;
import org.cloudbus.cloudsim.Log;
import org.cloudbus.cloudsim.lists.CloudletList;

/**
* FCFS Task scheduling
* @author
* Canul, Cindy
* Kumul, Cristian
* Peraza, Jonathan
*/
public class RoundRobinBroker extends DatacenterBroker{

public RoundRobinBroker(String name) throws Exception {
super(name);
}

public int get_quantum(){
int numTareas = cloudletList.size();
int resultado = 0;

for(int i = 0; i < numTareas; i++){
resultado += cloudletList.get(i).getActualCPUTime();
}

return resultado /= numTareas;
}

public void scheduleTasksToVms(){
int numTareas = cloudletList.size();
int numVirtualM = vmList.size();

Log.printLine("Round Robin Scheduler");
int index = 0;
int i = 0;
while( index < numTareas ){
long capacidad = vmList.get(i).getBw();
long costo_actual = 0;
long currentCost = (costo_actual + cloudletList.get(index).getCloudletLength()) / 100;
while (currentCost <= capacidad && index < numTareas) {
costo_actual += cloudletList.get(index).getCloudletLength();
bindCloudletToVm(index, i);
System.out.println("Task" + cloudletList.get(index).getCloudletId() + " is bound with VM" + vmList.get(i).getId() + " tam: " + cloudletList.get(index).getCloudletLength());
index++;
if(index >= numTareas) {
//Si terminamos de asignar todas las tareas
break;
}

}

i++;

if(i >= numVirtualM) {
//Reiniciamos la ronda de asignacion a la maquina 0
i = 0;
}
}
}

}

\end{lstlisting}


\subsection*{\textit{Heuristics Broker}}
\begin{lstlisting}
package Heuristics;

import PSO.PSOResult;
import PSO.PSOUtilities;
import PSO.Particle;
import PSO.Swarm;

import org.cloudbus.cloudsim.DatacenterBroker;

import java.util.List;

/**
* FCFS Task scheduling
* @author
* Canul, Cindy
* Kumul, Cristian
* Peraza, Jonathan
*/
public class HeuristicBroker extends DatacenterBroker {

public HeuristicBroker(String name) throws Exception {
super(name);
}

public void scheduleTaskstoVms(List<Particle> particleList) {

// Se crea el objeto Swarm
Swarm swarm = new Swarm();
// las tareas (particulas) son enviadas para evaluar
PSOResult result = swarm.init(particleList, vmList, 10, 2);

int cont = 0;
//Se realiza la iteracion hasta que todas las tareas esten calendariadas
while (cont < particleList.size()) {
// se asigna la mejor particula a la maquina virtual encontrada con PSO
bindCloudletToVm(result.getParticle().getCloudletId(), result.getVmId());
particleList.get(result.getParticle().getCloudletId()).scheduled = true;
System.out.println("Task" + result.getParticle().getCloudletId() + " is bound with VM" + result.getVmId() + "tam: " + result.getParticle().getCloudletFileSize());
cont++;
//Se ejecuta PSO con la lista actualizada
result = swarm.init(particleList, vmList, 10, 2);

}

}
}

\end{lstlisting}

\newpage

\section*{\textit{Metaheurística PSO}}
\addcontentsline{toc}{section}{\textit{Metaheurística PSO}}




\subsection*{\textit{Partícula}}
\begin{lstlisting}
package PSO;

import org.cloudbus.cloudsim.Cloudlet;
import org.cloudbus.cloudsim.UtilizationModel;


/**
* Created by Jake The Dog on 28/10/15.
*/
public class Particle extends Cloudlet {

/**
* Allocates a new Cloudlet object. The Cloudlet length, input and output file sizes should be
* greater than or equal to 1. By default this constructor sets the history of this object.
*
* @param cloudletId          the unique ID of this Cloudlet
* @param cloudletLength      the length or size (in MI) of this cloudlet to be executed in a
* PowerDatacenter
* @param pesNumber           the pes number
* @param cloudletFileSize    the file size (in byte) of this cloudlet <tt>BEFORE</tt> submitting
* to a PowerDatacenter
* @param cloudletOutputSize  the file size (in byte) of this cloudlet <tt>AFTER</tt> finish
* executing by a PowerDatacenter
* @param utilizationModelCpu the utilization model cpu
* @param utilizationModelRam the utilization model ram
* @param utilizationModelBw  the utilization model bw
* @pre cloudletID >= 0
* @pre cloudletLength >= 0.0
* @pre cloudletFileSize >= 1
* @pre cloudletOutputSize >= 1
* @post $none
*/

public boolean scheduled = false;

public Particle(int cloudletId, long cloudletLength, int pesNumber, long cloudletFileSize, long cloudletOutputSize, UtilizationModel utilizationModelCpu, UtilizationModel utilizationModelRam, UtilizationModel utilizationModelBw) {
super(cloudletId, cloudletLength, pesNumber, cloudletFileSize, cloudletOutputSize, utilizationModelCpu, utilizationModelRam, utilizationModelBw);
}

public void initialise() {


}


/* Extends attributes and methods */

/**
* the velocity of cloudlet particle is denoted by :
* v[i][k+1] = (inertia weight)v[i][k] + (acceleration coefficient 1)Rand(0,1) * (pbest - x[i][k]) + (acceleration coefficient 2)Rand(0,1) * (gbest - x[i][k])
* <p>
* -----------------------------------
*/
private int velocity;
private int position;
private double pbest;
private double gbest;

private int pbestPosition;
private int gbestPosition;

private int bestVm;

/**
* coefficients
* W = inertia weight
* C = acceleration coefficients
* rand = rand number between 0,1
*/
private int w = 1;
private int c1 = 1;
private int c2 = 1;

public static int getRandom() {
return ((int) (Math.random() * (1)));
}

public static int getRandomInteger(int maximun, int minimum) {
return ((int) (Math.random() * (maximun - minimum))) + minimum;
}

public void initParticle(int vms, int velocityLimit) {
this.position = getRandomInteger(vms, 0);
this.velocity = getRandomInteger(velocityLimit, 1);
this.pbest = Double.POSITIVE_INFINITY;
}

public int getVelocity() {
return velocity;
}

public void setVelocity(int gbest) {
c1 = getRandomInteger(2, 1);
c2 = getRandomInteger(2, 1);

velocity = (w * position) + ((c1 * getRandom()) * (this.pbestPosition - this.position)) + ((c2 * getRandom()) * (gbest - this.position));
}

public int getPosition() {
return this.position;
}

public void setPosition(int velocity) {
this.position = this.position + velocity;
}


public void updatePbest(double fit) {
//si el fit es menor que el pbest se actualiza
if (fit < this.pbest) {
this.pbest = fit;
}
}

public void setPbest(double fit) {
this.pbest = fit;
}

public double getPbest() {
return this.pbest;
}

public void setBestVm(int x) {
this.bestVm = x;
}

public int getBestVm() {
return this.bestVm;
}


}

\end{lstlisting}

\subsection*{\textit{FitnessFunction}}
\begin{lstlisting}
package PSO;

import PSO.Particle;
import org.cloudbus.cloudsim.Host;
import org.cloudbus.cloudsim.Vm;
import org.cloudbus.cloudsim.Cloudlet;
import common.CloudletUtilities;

import java.util.List;
/**
* FCFS Task scheduling
* @author
* Canul, Cindy
* Kumul, Cristian
* Peraza, Jonathan
*/

public class FitnessFunction {

public double evaluate(Cloudlet particle, Vm vm) {
CloudletUtilities utilities = new CloudletUtilities();

return utilities.getProcessingCostBefore(vm, particle);
}

}

\end{lstlisting}



\subsection*{\textit{Swarm Class}}
\begin{lstlisting}
package PSO;

import org.cloudbus.cloudsim.Vm;

import java.util.List;

/**
* Created by Jake The Dog on 16/11/02.
*/
public class Swarm {

public static int getRandomInteger(int maximun, int minimum) {
return ((int) (Math.random() * (maximun - minimum))) + minimum;
}

public static void initialize(List<Particle> list, int vms, int velocityLimit) {
for (int i = 0; i < list.size(); i++) {
list.get(i).initParticle(vms, velocityLimit);
}
}

public static Particle getBestParticle(List<Particle> list) {
int bIndex = 0;
double best = Double.POSITIVE_INFINITY;
for (int i = 0; i < list.size(); i++) {
if (list.get(i).getPbest() < best) {
best = list.get(i).getPbest();
bIndex = i;

}
}
return list.get(bIndex);
}

public static PSOResult init(List<Particle> particles, List<? extends Vm> vms, int iteration, int velocityLimit) {
double gbest = 0;
int gBestPosition = 0;
Particle gBestParticle;
PSOResult result = new PSOResult();
// Se inicia de manera aleatoria la posici\'on y la velocidad de cada particula
initialize(particles, vms.size(), velocityLimit);
//Criterio de terminaci\'0n de PSO
for (int x = 0; x < iteration; x++) {


//Para cada part\'icula, calcular su fitness value
FitnessFunction fitness = new FitnessFunction();
for (int i = 0; i < particles.size(); i++) {
if (!particles.get(i).scheduled) {
double fitnessValue = fitness.evaluate(particles.get(i), vms.get(particles.get(i).getPosition() % vms.size()));
//Si el acutal fitnessValue es mejor que el previo pbest, colocamos el fitnesValue como nuevo pbest
if (fitnessValue < particles.get(i).getPbest()) {
particles.get(i).setPbest(fitnessValue);
particles.get(i).setBestVm(particles.get(i).getPosition() % vms.size());
}
}
}
// se selecciona la mejor particula como gbest
gBestParticle = getBestParticle(particles);
gbest = gBestParticle.getPbest();
gBestPosition = gBestParticle.getPosition();

//Actualizamos la velocidad y la posicion de cada particula
result.setParticle(gBestParticle);
result.setVmId(gBestParticle.getBestVm());
result.setCost(gbest);
for (int i = 0; i < particles.size(); i++) {

particles.get(i).setVelocity(gBestPosition);
int velocity = particles.get(i).getVelocity();
particles.get(i).setPosition(velocity);
}
}


return result;


}
}

\end{lstlisting}



\subsection*{\textit{PSOResult Class}}
\begin{lstlisting}
package PSO;

import org.cloudbus.cloudsim.Vm;

/**
* Canul, Cindy
* Kumul, Cristian
* Peraza, Jonathan
*/
public class PSOResult {

private double cost;
private int vmId;
private Particle particle;

public PSOResult() {
this.cost = 0.0;
this.vmId = 0;
this.particle = null;
}

public void setCost(double x) {
this.cost = x;
}

public double getCost() {
return this.cost;
}

public void setVmId(int x) {
this.vmId = x;
}

public int getVmId() {
return this.vmId;
}

public void setParticle(Particle particle) {
this.particle = particle;
}

public Particle getParticle() {
return this.particle;
}
}

\end{lstlisting}


\subsection*{\textit{PSOUtilities Class}}
\begin{lstlisting}
package PSO;

import org.cloudbus.cloudsim.Cloudlet;

import java.util.LinkedList;
import java.util.List;

/**
* @author
* Canul, Cindy
* Kumul, Cristian
* Peraza, Jonathan
*/
public class PSOUtilities {

public List<Particle> parseParticles(List<? extends Cloudlet> listCloudlet) {

List<Particle> particles = new LinkedList<Particle>();

for (int i = 0; i < listCloudlet.size(); i++) {
Cloudlet cloudlet = listCloudlet.get(i);
Particle particle = new Particle(cloudlet.getCloudletId(), cloudlet.getCloudletLength(), cloudlet.getNumberOfPes(), cloudlet.getCloudletFileSize(), cloudlet.getCloudletOutputSize(), cloudlet.getUtilizationModelCpu(), cloudlet.getUtilizationModelRam(), cloudlet.getUtilizationModelBw());

particles.add(particle);
}

return particles;
}

public List<Cloudlet> particles2Cloudlets(List<Particle> list) {

List<Cloudlet> cloudlets = new LinkedList<Cloudlet>();

for (int i = 0; i < list.size(); i++) {
Particle particle = list.get(i);
Cloudlet cloudlet = new Cloudlet(particle.getCloudletId(), particle.getCloudletLength(), particle.getNumberOfPes(), particle.getCloudletFileSize(), particle.getCloudletOutputSize(), particle.getUtilizationModelCpu(), particle.getUtilizationModelRam(), particle.getUtilizationModelBw());
cloudlet.setUserId(particle.getUserId());
cloudlets.add(cloudlet);
}
return cloudlets;
}
}

\end{lstlisting}