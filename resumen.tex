%% Los cap\'itulos inician con \chapter{T\'itulo}, estos aparecen numerados y
%% se incluyen en el \'indice general.
%%
%% Recuerda que aqu\'i ya puedes escribir acentos como: \'a, \'e, \'i, etc.
%% La letra n con tilde es: \'n.
\chapter*{Resumen}
\addcontentsline{toc}{chapter}{Resumen}



El c\'omputo en la nube es una tecnolog\'ia prometedora como plataforma del futuro, permite el acceso a recursos de  \textit{hardware} de manera remota. Dentro del funcionamiento de los servicios es necesario optimizar los recursos en el centro de datos y para esto se utilizan los esquemas de calendarizaci\'on. La complejidad de la administraci\'on de los recursos y la calendarizaci\'on incrementa con el n\'umero de tareas, pues es un problema NP-dif\'icil, debido a las distintas caracter\'isticas de cada \textit{host} y lo heterog\'eneas que son las tareas.
En este proyecto se estudiaron y compararon cuatro esquemas de calendarizaci\'on para proponer una mejora en la optimizaci\'on de recursos en un sistema \textit{ERP} en la nube, para ello se utilizaron dos m\'etricas; costo de procesamiento y tiempo de ejecuci\'on de las tareas.


%%
%%\section*{Marco te\\'orico}
%%\addcontentsline{toc}{section}{Marco te\\'orico}
%%\subsection*{Cloud computing}
%%\addcontentsline{toc}{subsection}{Cloud computing}

