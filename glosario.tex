\lhead{\textit{Glosario}}
\chead{}
\rhead{\thepage}

\chapter*{Glosario}
\addcontentsline{toc}{chapter}{Glosario}
\section*{A}
\addcontentsline{toc}{section}{A}
\begin{itemize}
	\item \textbf{Amazon:} Compañ\'ia de comercio electr\'onico y servicios de computaci\'on en la nube.
	
\end{itemize}

\section*{C}
\addcontentsline{toc}{section}{C}

\begin{itemize}
	\item \textbf{Cloud Computing:} c\'omputo en la nube, tiene como objetivo ofrecer servicios a trav\'es de Internet.
	\item \textbf{Cloudlet:} Es una representaci\'on de tareas en Cloudsim.
	\item \textbf{CloudSim:} Framework para el modelado y simulaci\'on de la infraestructura y los servicios de Cloud Computing.
	\item \textbf{Cluster:} Conjunto de computadoras unidas entre s\'i normalmente por una red de alta velocidad y que se comportan como si fuese una \'unica computadora.
	
		
	
\end{itemize}
\section*{E}
\addcontentsline{toc}{section}{E}
\begin{itemize}
	\item \textbf{ERP:} Planificador de recursos empresariales, son sistemas inform\'aticos destinados a la administraci\'on de recursos de una organizaci\'on.
	
	
\end{itemize}


\section*{F}
\addcontentsline{toc}{section}{F}
\begin{itemize}
	\item \textbf{Framework:} Marco de trabajo, es un conjunto de clases que implementa todos los servicios comunes de un cierto tipo de aplicaci\'on.
\end{itemize}


\section*{P}
\addcontentsline{toc}{section}{P}
\begin{itemize}
	\item \textbf{Pruebas de software/hardware:} t\'ecnicas cuyo objetivo es evaluar la calidad del software/hardware.

\end{itemize}




