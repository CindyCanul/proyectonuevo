\lhead{\textit{Conclusiones}}
\chead{}
\rhead{\thepage}

\chapter*{Conclusiones}
\addcontentsline{toc}{chapter}{Conclusiones}
%\section*{Recomendaciones}
%\addcontentsline{toc}{section}{Recomendaciones}

En éste proyecto, se presentó una serie de esquemas de distribución con implementaciones sencillas para seleccionar \textit{Min-Min} y complementarlo con una metaheurística  llamada \textit{Particle Swarm Optimization (PSO)}. Se usó ésta técnica con el objetivo de minimizar el costo de procesamiento y el tiempo de ejecución en un centro de datos con un entorno en la nube. Se encontró que el esquema propuesto reduce el costo de procesamiento en \%50 en comparación con una calendarización sin procesamiento previo. Sin embargo se determinó que el mapeo de cada partícula en los recursos del centro de datos, tiene un costo en tiempo de ejecución, ya que no se vio mejoría. Por lo tanto el costo de procesamiento de una tarea en un recurso del centro de datos (en éste caso las máquinas virtuales) es inversamente proporcional al tiempo que se toma en resolver la tarea.
PSO encuentra los recursos del centro de datos en donde las tareas a procesar tendrán un menor costo. Ésta heurística tiene un carácter genérico, es decir puede ser usado para cualquier número de tareas en el centro de datos y los recursos en el centro de datos pueden tener cualquier dimensión. Ya que simplemente se incrementa el número de partículas y el espacio de búsqueda se sería mayor.
Como trabajo a futuro, sería interesante analizar el comportamiento del costo de procesamiento con variaciones en las políticas de alojamiento de las máquinas virtuales.



